\documentclass[paper=A4, pagesize]{scrreprt}

\usepackage[T1]{fontenc}
\usepackage[utf8x]{inputenc}
\usepackage[ngerman]{babel}
\usepackage{fixltx2e}

\usepackage[nonamebreak]{natbib}


\usepackage[usenames,dvipsnames]{xcolor}
\usepackage{ae}
\usepackage{listings}
\usepackage {ellipsis, ragged2e}
\usepackage[babel]{csquotes}
\usepackage[pdfborder={0 0 0}, colorlinks=true, urlcolor=blue, linkcolor=Black, breaklinks]{hyperref}
\usepackage[hyphenbreaks]{breakurl}
\usepackage{graphicx}
\usepackage{nameref}

\setlength{\parskip}{0.3cm}
\linespread {1.25}

\usepackage{xcolor}
\definecolor{dark-blue}{rgb}{0.15,0.15,0.4}
\definecolor{medium-blue}{rgb}{0,0,0.5}
\hypersetup{
    colorlinks, linkcolor={dark-blue},
    citecolor={dark-blue}, urlcolor={medium-blue}
}

\begin{document}

\bibliographystyle {alpha}

\nocite{*}

\begin{titlepage}
\vspace*{2cm}

\begin{center}
\Huge
Unplagged\\
---\\
\large
The Plagiarism Detection Cockpit\\
Developers Manual\\
\vfill
Term paper for the master project I \\ Mentoring Teacher: Prof. Dr. Debora Weber-Wulff\\
\vspace{1cm}
\normalsize
Department Economics II\\
HTW Berlin -- University of Applied Sciences\\

\vspace{4mm}
Elsa Mahari (s0534217) \href{mailto:elsa.mahari@gmx.de}{\textless Elsa.Mahari@gmx.de\textgreater}\\
Heiko Stammel (s0534217) \href{mailto:heiko.stammel@googlemail.com}{\textless heiko.stammel@googlemail.com\textgreater}\\
Benjamin Oertel (s0522720) \href{mailto:benjamin.oertel@me.com}{\textless benjamin.oertel@me.com\textgreater}\\
Dominik Horb (s0534217) \href{mailto:dominik.horb@googlemail.com}{\textless dominik.horb@googlemail.com\textgreater}\\
Tien Nguyen (s0534217) \href{mailto:idontwant2missathing@yahoo.com}{\textless idontwant2missathing@yahoo.com\textgreater}\\


\end{center}
\end{titlepage}

% arabic page numbering should start after the toc, so we start with roman
\pagenumbering{Roman}

\begingroup % to make sure not every item in the toc is weirdly colored
\hypersetup{linkcolor=black}
\tableofcontents
\endgroup

\clearpage % to ensure that the arabic numbering starts on the page after the toc
\pagenumbering{arabic}

\chapter{Introduction}

Even though the big media coverage and interest in plagiarism in Germany has very much subsided 
after Minister Guttenberg had to resign, because of the plagiarisms found in his doctoral thesis\citep{Google2012}, 
the initial idea for the 
creation of the \enquote{Unplagged} project, whose development approach will be described here, can be found in this 
very case of plagiarism. Related to it were the formation of the
\href{http://de.guttenplag.wikia.com/wiki/GuttenPlag\_Wiki}{GuttenPlag} and it's descendent 
\href{http://de.vroniplag.wikia.com/wiki/Home}{VroniPlag}. Both are Wiki-based communities that are collaboratively 
discovering and collecting plagiarism in their respective cases and are kind of the role models for the way the
Unplagged system is developed.

The initial project idea and context were provided by our professor Dr. Debora Weber-Wulff and the two term master project
every media informatics student at the \href{http://htw-berlin.de/}{HTW-Berlin} has to take. As professor Weber-Wulff is 
a well known expert in Germany on the
topic of plagiarism,
does research in this field for over ten years\citep{Spiegel-Online2011} and is actively involved in the VroniPlag
community, she came up with the idea to build a dedicated system --- a \enquote{Plagiarism Detection Cockpit}\citep{Weber-Wulff2011}, 
that is modeled after the experiences that were made
with the workflow used in VroniPlag and GuttenPlag.


So, to put it in a catchy marketing phrase, here is what Unplagged aims to become: 

\begin{quote}
\textbf{Unplagged is a simple, web-based, collaborative system to help discovering, collecting and 
documenting plagiarism in scientific papers.}
\end{quote}

To make things a bit more conceivable, we also often refer to it as a mixture of a very specialised text editor, with a focus on 
comparing texts and marking 
passages and a modern project management tool like \href{http://www.redmine.org/}{Redmine} or 
\href{http://www.atlassian.com/JIRA}{JIRA}, 
to manage the collaborative aspects of the system. The big distinction we make to other plagiarism software on the market is, 
that the approach is not to autodetect plagiarism, but focused on aiding the workflow of the users while  
searching for plagiarized
fragments inside a scientific paper, a homework assignment or any other kind of probable plagiarism.

This present document will be your handbook, if you want to get started helping in the development of this 
open source project, which is licensed under the \href{http://www.gnu.org/licenses/quick-guide-gplv3}{GNU GPLv3}.

\section{Chapter Overview}

One of the biggest problems we faced at the start was, that none of the team members had written a longer scientific
text than a bachelors
thesis and therefore the experience we got with actual scientific writing was very limited and very specific to the 
field of computer science. We understand the ethical problems, that come with the betrayal of 
good scientific practice of plagiators, but we simply can not relate easily to the amount of work that has to be put into 
a Ph D., or be as 
passionate about plagiarism as Prof. Weber-Wulff always is, because we never experienced it.

That is why we had a lot of catching up to do on the most important history behind VroniPlag, the different types
of plagiarism, different citation styles and the research Prof. Weber-Wulff and others had already done on systems that try to 
help finding plagiarism. The chapter \ref{chap:plagOverview}, \nameref{chap:plagOverview}, will
give a brief overview of the most important topics to get you up to speed with the domain of the software, if you are
not already familiar with it.

Although we are using agile methods for the development process, the chapter \nameref{chap:systemRequirements} will give
a more classical collection and description of the parts of the system, that already exist or that we identified as 
necessary parts of Unplaggedhow we understood the requirements of the 
VroniPlag workflow.

If you know all those things already and simply want to get started working and coding, you should probably jump
to \nameref{chap:developingUnplagged}. This chapter will give insights into the project workflow, the basic installation
steps and all necessary tools for you as a developer.

%In the appendices..

\chapter{A Plagiarism Primer}\label{chap:plagOverview}

Here we will introduce you to the topics we encountered and researched in the beginning of the Unplagged project.

\section{Plagiarism definition}

% Das können wir meiner Meinung nach so nicht lassen, wir können doch nicht die Definition von Plagiat abschreiben..

You will probably have learned as a child, that stealing material goods is something that will get you arrested sooner or
later. Somehow many people make a huge difference when it comes to intellectual property, maybe due to the fact that 
stealing ideas or texts is much easier to do and much harder to catch.

Nonetheless, this is something that would fall under copyright laws and is also a crime, which could get prosecuted.

Here is what \citet{PlagiarismDotOrg} thinks about plagiarism:

\begin{quote}\enquote{
Many people think of plagiarism as copying another's work, or borrowing someone else's original ideas. But terms like 
\enquote{copying} and \enquote{borrowing} can disguise the seriousness of the offense: [...] In other words, plagiarism 
is an act of fraud. It involves both stealing someone else's work and lying about it afterward.
}
\end{quote}

The Community Standard for Undergraduates of the \citet{DukeSite} University  shows examples for the difference between intentional and unintentional Plagiarism:

\textbf{Intentional Plagiarism}


\begin{itemize}
\item Purchasing a pre-written paper (either by mail or electronically).
\item    Letting someone else write part or all of a paper for you.
\item    Paying someone else to write part or all of a paper for you.
\item    Submitting as your own someone else's unpublished work (including a computer program or algorithm), either with or without permission.
\item    Submitting as your own, work done jointly by a group in which you may have participated.
\item    Submitting work done by you, but for another class or another purpose without documenting that it was previously used.
\item    Creating phony citations.
\end{itemize}

\textbf{Unintentional Plagiarism}
\begin{itemize}
\item Failure to cite a source that is not common knowledge.
\item Failure to "quote" or block quote author's exact words, even if documented.
\item Failure to put a paraphrase in your own words, even if documented.
\item Failure to put a summary in your own words, even if documented.
\item Failure to be loyal to a source.
\end{itemize}




	
	
\section{Basic Classification of Plagiarisms}

In this part, we will try to cover common classifications of plagiarism, but at first we want to figure out the purpose of 
the classification. 

There are many reasons explaining why people plagiarize. Generally those include, that they didn't have the time,
energy or the ability to do the work by themselves or that they try to steal other’s work on purpose with the hope that 
others
will not recognize it. This kind of plagiarism is done intentionally and named \textit{deliberate 
plagiarism}\citep{UEfAP}.

Another kind is \textit{accidental plagiarism}. This occurs when texts from some sources are copied or rephrased, but no 
reference is given. The reason is that the writer didn't know that it is a plagiarism because of \enquote{carelessness or  
lack of skill}\citep{UEfAP} while writing.

Anyway, it is important to know, that working with the source carelessly may cause plagiarism. Classification 
of plagiarism is also a good way to help distinguish typical types of plagiarism, so that people are aware and able 
to reference the sources carefully and to avoid plagiarizing.

Classification of plagiarism is also good for professors or the plagiarism detection community such as VroniPlag, because
it gives them a common terminology, which enables them to communicate faster and more efficiently. 
It also helps them to realize what category and how many percent the text is plagiarized 
if there is any suspicion, so that statistical data can be created.

% I don't understand this sentence at all..
%So the classification is about the way to detect plagiarism while conducting a citation. 
But what is a citation, and 
what kinds of citation are there? Understanding of citations and the way to cite is an important thing in order to 
avoid and detect plagiarism.

\subsection{Definition of Citation}

\begin{quote}\enquote{A citation is a credit or reference to another document or source}\\ \citep{wiki:Citation}\end{quote}

According to this definition, citation is your reference to a source of information, which was generated by someone 
and should be indicated properly when this source is used in your work.

Based on \citet{Wiredprof2010}, there are 3 forms of citation:

\begin{description}
\item[Direct quotation:] \hfill \\
That is an exact word-to-word copy from one \textit{source}. In this case, in order to cite you have to mark 
the text with quotation marks and indicate a  parenthetical referencing, which includes the \textit{book’s name and the page 
number} where you found the source.
\item[Paraphrase:] \hfill \\
That is your own explanation of someones idea. Many people think it is not a plagiarism because the text 
is written with your own words. But it is still a plagiarism because basically that is not your idea or your opinion, 
but the author’s himself. In this case of citation, some keywords are still kept in the work. Therefore the 
parenthetical reference must be given.
\item[Summary:] \hfill \\
That is the same as paraphrase, but this kind of citation is likely a \textit{summary} of the text. In this case you 
have to give the parenthetical referencing as well.
\end{description}

Now we can see that citation has a lot of forms. You are free to use the source but you have to make sure to cite 
sources properly 
or you are applying a plagiarism.

After understanding what citation is and its form, now we can try to classify the plagiarism. The criteria which we 
choose for classification are based on the VroniPlag’s plagiarism categories, because Unplagged is mostly based 
on the workflow of the VroniPlag community. 

By VroniPlag, it does not matter what citation standard system is used in the work, but it is important to know how the 
citation is documented and if the reference is given properly.

\subsubsection{VroniPlag’s classification of plagiarism}\label{sec:classification}

According to VroniPlag\citep{} there are the following plagiarism categories. 
The categories are originally written in German, and the translation or the similar category in English is written 
right after the german expression.

\begin{description}
\item[Komplettplagiat/Copy \& Paste\citep{one12}:] \hfill \\
The name of this category already indicates how the text is plagiarized. The 
plagiariser just copies from the source exactly word-by-word and does not leave any proper reference to the source 
intentionally. Some is too lazy so that they just copy existing mistakes, formats etc... without checking, which 
later can become a proof of plagiarism.

The category is differentiated by the way of conducting a citation.
Source is not cited: the original text is completely copied but the source reference is not given intentionally.
Source is cited but not completely: reference is given but not correctly.  

\item[Verschleierung/Paraphrasing\citep{PlagiarismSearch}:]  \hfill \\
The texts from different sources are rephrased and mixed together. The plagiariser tries 
to hide his stealing by changing some word orders, replacing words with synonyms … The source is not given with the hope, 
that the text is considerably generated by the plagiariser himself and therefore the plagiarism will not be detected.

\item[Übersetzungsplagiat/translations\citep{one12}:] \hfill \\
This kind of plagiarism occurs when the sources in foreign languages are translated. 
The plagiariser pretend that this translated text is his own work. Sources are oft not given properly. This is a 
well-liked way in researching because there are many translation tools which can be found easily in Internet such as 
Babelfish, Google Translate … and  it is not easy to detect the plagiarism with existing plagiarism detection tools.

\item[Alibi-Fußnote/The forgotten Footnote\citep{PlagiarismSearch}:] \hfill \\
in this category, the source is cited but the real location of the source is 
hidden.  

\item[Bauernopfer und. Verschärftes Bauernopfer:] \hfill \\
The source reference is embedded in footnote part but it redirects to 
another text part of the source which has no relation with the plagiarized text.
\end{description}

\paragraph{Other categories}


\begin{description}
\item[Halbsatzflickerei/The Labor of Laziness\citep{PlagiarismSearch}:] \hfill \\
The plagiariser takes sentences, or just parts of sentence from different 
sources, tries to reword them and mixes all so that they look fit together. Source reference is not or not correctly 
given.

\item[Shake \& Paste\citep{one12}:] \hfill \\
The sentence or paragraph from one source is mixed with one from another source. This 
plagiarism can be detected by changes of writing styles. Source reference is also not given properly.
\end{description}

\section{How to detect plagiarism}
Plagiarism detection means a lot of effort and hard work, because a lot of documents, books and papers must be found, scanned and compared line for line. Often it is really complicated to find available sources in libraries or scientific database systems and plenty of pages must be copied by hand. 
During the last ten years a lot of software companies came into market to automatize this process and developed algorithms to find plagiarisms automatically. These systems are called "Plagiarism detection systems (PDS) ". 
There are two different approaches for plagiarism detection systems to identify plagiarism in text documents:
\begin{enumerate}
\item \textbf{Corpus based analysis}\\
means to "compare suspicious documents against a set of potential original documents" \citet{PAN:2007} to find similar text passages.
\item \textbf{Intrinsic analysis}\\
"identifies potentially plagiarized passages by analyzing the suspicious document with respekt to changes in writing style". \citep{PAN:2007}
\end{enumerate}
These computer assisted detection systems alone are not appropriate to find all plagiarisms without human judgement.  

 \begin{figure}[!h]
  \centering
  \fbox{
    \includegraphics[width=0.97\textwidth]{images/4-steps.png}
  }
  \caption{4-stage plagiarism detection process}
  \label{fig:4-stage plagiarism detection process}
\end{figure}

We found out that the most practical way is to combine both approaches - the first step is to use the computer-based detection to find similarities between the suspicious documents and the original papers. The second step is to examine the results, validate them and continue the search in a deeper level.\citep{PI:2001} (see figure \ref{fig:4-stage plagiarism detection process}) 



\section{Commercial software systems} 

 
 \begin{figure}[!h]
  \centering
  \fbox{
    \includegraphics[width=0.97\textwidth]{images/software_systems/overview.png}
  }
  \caption{Overview of Commercial Detection Systems}
  \label{fig:overview_systems}
\end{figure}



Several computer companies developed commercial software systems to facilitate the detection of plagiarism. They offer different terms of pricing and online as a service or offline programs. 
The software system compare digital content from the internet or different types of databases. 
They seek for similarities, report suspicious parts and try to answer the question if the present text is a plagiarism or not.

This chapter covers parts of the results of the big "Plagiarism Detection System Test 2010". 
The following benchmarking was done by Prof. Debora Weber-Wulff at the University of Applied Sciences HTW Berlin and her plagiarism team in 2010. \citep{PlagiatTeam} 
The entire benchmarking includes 26 of the 47 available systems on the market and gives an overview of the strengths and weaknesses of these systems in finding plagiarism.

We used the top 5 "partially useful" software systems in this  benchmarking to overview the market and find the best usable features for our own system to simplify the daily work of plagiarism finders. 
\citep{PlagiatTeam} 






\newpage



\subsection{PlagAware} 
 \begin{figure}[!h]
  \centering
  \fbox{
    \includegraphics[width=0.97\textwidth]{images/software_systems/plagaware_1.png}
  }
  \caption{Plagaware Website}
  \label{fig:plagawareWebsite}
\end{figure}


\textbf{Plagaware Business Promotion}
\begin{quote}
"PlagAware is an online-service, which offers services around the topics Searching, finding, analysing and tracing of plagiarisms. The central element of PlagAware is a search engine, which is specialised in detecting identical contents of given texts. Contrary to the plagiarism scanning with classical search engines, the places of finding are not directly transferred to the user, but analysed on the rate and the type of analogy, before a message to the user is written. By this the differing result reports of PlagAware allow to recognise very fast the percentage and the distribution of the copied text contents, thus permitting an efficient and secure rating of a possible plagiarism."  
\end{quote}\citep{PlagawareTest}


PlagAware is software company from Ulm in Germany. The website is online for nearly 5 years and is the top-ranked system in the HTW "Plagiarism Detection System Test" 2010, but in fact it still detect only 61,11\% of the plagiarism cases. 
Although PlagAware "produces excellent documentation of the plagiarism found, highlighting the commonalities in a side-by-side presentation. However, its usefulness at university is limited, as each file must be uploaded individually - no ZIP file or student-submission is possible. The system was not designed to be used in a university setting, but rather to find plagiarisms of online texts, which is important for sites trying to optimize their search machine ranking, as plagiarism will contribute to downranking." \citep{PlagawareTest}

Figure Overview shows an overview of all fragments and the results of the plagiarism detection. In figure Side-by-side shows a nice side-by-side fragment view, where all found plagiarisms are shown with different colors. Anomalies in the text are highlighted and the barcode-view is available.
 
 \begin{figure}[!h]
  \centering
  \fbox{
    \includegraphics[width=0.97\textwidth]{images/software_systems/plagaware_2.png}
  }
  \caption{Plagaware overview}
  \label{fig:plagawareoverview}
\end{figure}


 \begin{figure}[!h]
  \centering
  \fbox{
    \includegraphics[width=0.97\textwidth]{images/software_systems/plagaware_3.png}
  }
  \caption{Plagaware side-by-side view}
  \label{fig:plagaware_side_by_side}
\end{figure}


\newpage
\subsection*{PlagAware Costs}
PlagAware has four different payment-models:
\begin{enumerate}
\item \textbf{Free}\\
30 scans/month for free. Every additional scan costs 3,0 ct. 
\item \textbf{Light}\\
EUR 2,99/month. 150 scans/month included. Every additional scan costs 2,0 ct. Mininum term of 6 month.
\item \textbf{Standard}\\
EUR 7,49/month. 500 scans/month included. Every additional scan costs 1,5 ct. Mininum term of 6 month.
\item \textbf{Premium}\\
EUR 14,99/month. 1500 scans/month included. Every additional scan costs 1ct. Mininum term of 6 month.
\end{enumerate}\citep{PlagawareTest}



\newpage
\subsubsection{Turnitin} 
 \begin{figure}[!h]
  \centering
  \fbox{
    \includegraphics[width=0.97\textwidth]{images/software_systems/turnitin_1.png}
  }
  \caption{Turnitin Website}
  \label{fig:Turnitin Website}
\end{figure}

\textbf{Business Promotion}
\begin{quote}
"Our award-winning solution discourages plagiarism and facilitates rich, meaningful feedback that improves writing skills, promotes critical thinking, and streamlines grading."
\end{quote}
\enquote{Turnitin}\citep{Turnitin Business Promotion}\href{http://www.turnitin.com}{Turnitin}


Turnitin is a product by a company called iParadigms. It is a well-known US plagiarism software system and one of the most used plagiarism detection systems in the education sector. The website is online for nearly 13 years and the system is at the second position in the HTW "Plagiarism Detection System Test" 2010.  

The best results can be achieved with material which is already in the database. 

In the past the had a lot of problems to deal with umlauts, and having a complex setup. They improved a lot of parts especially the german translation. Still a big problem for european countries is the copyright policy of Turnitin. They still storing copies of user material in their database without a permission. \citep{TurnitinTest}

 \begin{figure}[!h]
  \centering
  \fbox{
    \includegraphics[width=0.97\textwidth]{images/software_systems/turnitin_3.png}
  }
  \caption{Turnitin - A lot of little matches can't be found, if the sensibility has not been raised.}
  \label{fig:Turnitin overview}
\end{figure}
 
In 2008 the system was placed at the 13th position. The reason of this change is that the others systems have gotten worse. Turnitin has still problems of flagging spam sites especially when this sites are not safe for work (e.g. site with pornography content) (see \ref{fig:TurnitinSpamExample}). This is a problem when students uses school or university computers.
On the other hand the search algorithm of Turnitin is storing sites in their database although they are still not exist.\citep{TurnitinTest}


\subsection*{Costs}
There are different license models for the education sector and the cost depends on the amount of users . 

 \begin{figure}[!h]
  \centering
  \fbox{
    \includegraphics[width=0.97\textwidth]{images/software_systems/turnitin_4.png}
  }
  \caption{Turnitin - A lot of spam-sites are reported. Not all sufficient to use at work. This is one of the harmless examples.}
  \label{fig:TurnitinSpamExample}
\end{figure}






\newpage

\subsection{Ephorus}

 \begin{figure}[!h]
  \centering
  \fbox{
    \includegraphics[width=0.97\textwidth]{images/software_systems/euphorus_1.png}
  }
  \caption{Ephorus Website}
  \label{fig:plagawareWebsite}
\end{figure}

\textbf{Business Promotion}
\begin{quote}
"Never search for plagiarism yourself again? An end to all irritations and qualitatively better papers? No problem. With Ephorus, you can prevent plagiarism with no extra effort. Moreover, with this anti-plagiarism market leader, you will be assured of the best service and the lowest prices. With Ephorus, teaching will be fun again! Would you like to try out Ephorus?
\end{quote}
\citep{EphorusTest}

The third position in the HTW "Plagiarism Detection System Test" 2010 is Euphorus. It's a plagiarism detection system from the netherlands and the website is online for nearly 8 years.
2007 it took the first place in the test, 2008 it was only position 8. Now they redisigned and reorganized the system and old problems were solved. The usability of reports and the whole handling of the system very good. (figure: report)But their still problems with umlauts and the european copyright problematic like in Turnitin. (figure: umlauts)


 \begin{figure}[!h]
  \centering
  \fbox{
    \includegraphics[width=0.97\textwidth]{images/software_systems/euphorus_3.png}
  }
  \caption{Ephorus report - gives a great overview of the results}
  \label{fig:Ephorus_report}
\end{figure}




\subsection*{Ephorus costs}
Not stated.

 \begin{figure}[!h]
  \centering
  \fbox{
    \includegraphics[width=0.97\textwidth]{images/software_systems/euphorus_5.png}
  }
  \caption{Ephorus Problem with umlauts}
  \label{fig:Euphorus_umlauts}
\end{figure}













\newpage
\subsection{PlagScan} 

 \begin{figure}[!h]
  \centering
  \fbox{
    \includegraphics[width=0.97\textwidth]{images/software_systems/plagscan_1.png}
  }
  \caption{Plagscan Website}
  \label{fig:plagawareWebsite}
\end{figure}


\textbf{Business Promotion}
\begin{quote}
\textbf{PlagScan stands for professionalism}
\begin{itemize}
\item All documents are treated 100\% confidential
\item    You control whether your document is checked against others, or not
\item    Integration via API in your existing CMS or learning management system possible
\end{itemize}
\textbf{Plagiarism check as easy as pie: PlagScan}
\begin{itemize}
\item    Annotations directly in the document, check without additional work
\item    No installation - complete functionality in every browser
 \item   All popular formats can be processed
\end{itemize}
\textbf{Save time with PlagScan}
\begin{itemize}
\item    Check several documents in parallel.
\item    Fully automated document analysis.
 \item   No use of your resources, all computation is carried out on our servers.
\end{itemize}

\end{quote}
\enquote{Plagscan Business Promotion}\citep{PlagscanTest}

Plagscan is a software company from Mainz, Germany. It placed at position number 4 in the HTW "Plagiarism Detection System Test" 2010. The website is online for 3 years and in the preview check 2008 it came to the 10th position.
As a user you have to buy "Plag Points" (PP). One test costs 1 PP per 100 words. 
The administrator sets up users and assigns them points for use.
There are three different kinds of reports - a list of possible sources with links to click on, the submitted document with the suspicious areas linked to a possible source, and a docx file with the sources in comments.
There's no side-by-side presentation, so it's not possible to compare the fragments.
Although there are still problems, PlagScan was first place in usability, but only 8th place in overall effectiveness with only 60\% of the points awarded for finding plagiarisms.\citep{PlagscanTest}


 \begin{figure}[!h]
  \centering
  \fbox{
    \includegraphics[width=0.97\textwidth]{images/software_systems/plagscan_2.png}
  }
  \caption{Plagscan report is clear and tidy.}
  \label{fig:plagaware report}
\end{figure}


 \begin{figure}[!h]
  \centering
  \fbox{
    \includegraphics[width=0.97\textwidth]{images/software_systems/plagscan_3.png}
  }
  \caption{Plagscan reports are not self-explanatory.}
  \label{fig:plagaware report 2}
\end{figure}

\subsubsection{Plagscan costs}
PlagScan has four different payment-models without a contract:
\begin{enumerate}
\item \textbf{9 Euro}\\
500 Plagpoint - 50.000 words - 200 Sites. 
\item \textbf{19 Euro}\\
1.250 Plagpoints - 125.000 words - 500 Sites. 
\item \textbf{29 Euro}\\
2.000 Plagpoints - 200.000 words - 800 Sites. .
\item \textbf{69 Euro}\\
5.000 Plagpoints - 500.000 words - 2.000 Sites. 
\end{enumerate}\citep{PlagscanTest}


\newpage
\subsection{Urkund}

 \begin{figure}[!h]
  \centering
  \fbox{
    \includegraphics[width=0.97\textwidth]{images/software_systems/urkund_1.png}
  }
  \caption{Urkund Website}
  \label{fig:Urkund Website}
\end{figure}

\textbf{Business Promotion}
\begin{quote}
"URKUND was born from the academic world. A team of teachers developed the idea of a web based service that would help them detect and deter plagiarism and URKUND was born in the fall of 2000. The problem of plagiarism received much attention in the media and more and more began realise the scope of the problem and the need of a tool to support the pedagogical work. URKUND continued to grow and develop over the years and came to be recognised as Sweden's foremost anti plagiarism service.

Today, URKUND is present in our neighbouring countries and continental Europe as well as the USA, Asia and the Middle East.

URKUND is a natural part of the educational work of the academic world today. Both faculty and students are aware of the immediate and long term benefits of our system."
\end{quote}\citep{UrkundTest}

Urkund is the last system in our comparison of partially useful systems. It ranked at the 5th position in the HTW "Plagiarism Detection System Test" 2010. The company is from sweden and started their business in 2000. In this test it ranked high in effectiveness but on the other side it's not easy to use. It has problems in the translation and after the redesign 2008 the usability was going worse.
Overall "the navigation is confusing, the layout at times catastrophic with texts overlapping fields,  the printed reports could be better, the error messages are cryptic, and the link descriptions are unclear."\citep{UrkundTest}


 \begin{figure}[!h]
  \centering
  \fbox{
    \includegraphics[width=0.97\textwidth]{images/software_systems/urkund_2.png}
  }
  \caption{Urkund List view}
  \label{fig:Urkund_list_view}
\end{figure}

 \begin{figure}[!h]
  \centering
  \fbox{
    \includegraphics[width=0.97\textwidth]{images/software_systems/urkund_3.png}
  }
  \caption{Urkund Report}
  \label{fig:urkund_report}
\end{figure}

\subsubsection{Urkund costs}
Not stated.


\newpage

\subsection{Resume commercial software systems}

Although over the years there are more software detection systems that claim to check text reliable if it's plagiarism or not, but the quality of the systems decreases. 
A big problem that some of the tested systems offer "ghostwriting".

There are a quite a few differences between the benchmarking in 2008 and 2010, specially the ability to detect plagiarism in text which dropped and switched words. The best systems only reached 70\%.

The "Plagiat Team" updated the test with short essays in german, english and japanese.
Also took aspects of design, language consistency, navigation and so on.
They  categorized the systems in partially useful, barely useful, and useless for university purposes. The best systems between 60 and 70\% effectiveness were PlagAware, Turnitin, Ephorus, PlagScan and Urkund. \citep{PlagiatTeam} 

The recommendation of  the \citet{PlagiatTeam} were that the focus should be on teaching students about plagiarism and how to avoid it instead of investing time in using software. 


\section{Vroni Plag}
VroniPlag is a wiki platform which many volunteers, who are ready to give their free time, their money or 
their books/resources etc. work on. They collaborate with each other in order to detect plagiarism in dissertations or
habilitations.



VroniPlag is named after the first case which they published. The case was the one of Veronica Saß.

Because it is a wiki page, which means open to all, every one could join in. If somebody is interested in a public 
case, he can feel free to edit the page without asking for an allowance. In the chat portal of this community, one could 
ask for more help or to take a look at the list of waiting fragments.

Before starting detection, there are some information that a collaborator might know.

\subsection{Plagiarism detection steps}
A suspicious case is called \textit{candidate case}. This case is still not public for all. If a user has detected some 
suspicious part of a dissertation, he/she may first go to Chat portal to indicate his/her suspicion. He/she has 
also to give at least one original text source as proof for it. If it is well reasoned that there is an existing 
plagiarism, he/she must see if there are enough collaborators, who are ready to spend their time to work with.

The candidate case has been given an anonymous name and not public until there is proof, that there is at 
least 10\% of the pages, in which plagiarized texts are found. After that the case will be published with the 
name of the author.

During the detection process, the candidate case will be divided into smaller fragments. The fragments will be checked 
carefully if there is a plagiarism found. If there is, a report is created which shows in which fragment the 
plagiarism is found. The original source is also correspondingly given.

After checking, fragment’s state is changed to \enquote{to be proofed.} That means, this fragment must be checked again  by a 
second inspector. The result will be classified in corresponding category (see \nameref{sec:classification}).

After all fragments are checked, an overview of detected plagiarism will be generated. The overview includes the following 
parts:

\subsubsection{Part 1:} 

The whole page numbers with two different colors. The page number includes also a link redirecting to corresponding 
fragment.

\begin{itemize}
\item Grey: page in which there is no plagiarism found or not checked yet.
\item Blue: page with plagiarised texts found. The link goes to the plagiarized text in comparison with  the source as well.
\end{itemize}


\begin{figure}[!h]
  \centering
    \includegraphics[width=0.95\textwidth]{images/vroni-pages.png}
  \caption{Source: \url{http://de.vroniplag.wikia.com/wiki/Lm}, 19/03/2012, 08:53}
  \label{fig:vroniPages}
\end{figure}


\subsubsection{Part 2:} 

The second part is a generated barcode label which performs the percent of plagiarism found in one page with 
different colors.

\begin{itemize}
\item Blue: pages which are not calculated in the dissertation such as Index, Appendix, literature list
\item Grey: suspiciously plagiarized
\item Black: verified that 100\% of the page is plagiarized
\item Brown: verified that more than 50\% of the page is plagiarized
\item Red: verified that more than 75\% of the page is plagiarized
\end{itemize}

\begin{figure}[!h]
  \centering
    \includegraphics[width=0.95\textwidth]{images/vroni-barcode.png}
  \caption{Source: \url{http://de.vroniplag.wikia.com/wiki/Lm}, 19/03/2012, 08:54}
  \label{fig:vroniBarcode}
\end{figure}



If there is more than 10\% of the whole pages plagiarized, the case will be public on wiki. Then a report will be sent 
to the university, where the dissertation is finished.

\subsection{Technical support}

Most of the work by VroniPlag is done by hand. For example collaborators could use Google to search for sources, 
or they borrow books from the library and scan the texts. Generally there is no special software to help detect 
plagiarism, but collaborator could feel free to choose some of the existing software to help work faster.


	 \begin{figure}[!h]
  \centering
  \fbox{
    \includegraphics[width=0.6\textwidth]{images/colors.png}
  }
  \caption{colors}
  \label{fig:colors}
\end{figure}

	


\chapter{System Requirements}

\section{Target Group}

\section{User roles}

\section{Basic functionalities}

\section{Document Parser}

\section{Detection Modes}

\section{Plugin Architecture}

\section{Use Cases}


\chapter{Developing Unplagged}\label{chap:developingUnplagged}

Coming from the \nameref{chap:systemRequirements} here we have yet another set of requirements for you, before we can
start with the actual description of the technologies used for development in the system. This time it's 
about what we believe will be helpful or sometimes even necessary for the development of Unplagged. 

First of all, the programming languages mostly used in Unplagged are PHP and JavaScript, both of which in conjuction
with a framework. Teaching programming languages is, as you probably can imagine well beyond the scope of this document,
but we will at least try to cover the most important concepts of the frameworks as they occur. 

The used frameworks are 
\href{http://jquery.com/}{jQuery} for Javascript and \href{http://framework.zend.com/docs/overview}{ZEND} for PHP 
respectively. jQuery is kind of the industry standard for unobtrusive scripting with about 50\% 
of the Top 10.000 websites using it according to \citet*{Trends} and the Zend framework is also well established and
brings a lot of features, that are useful to this project.

For most of the other topics, we will give you some (hopefully) helpful resources on the way, if it isn't covered 
thoroughly by us. But just to let you
know, here is a list of the buzzwords, er technologies that will be mentioned:

\begin{itemize}
\item CSS3
\item HTML5
\item Continuous Integration
\item Responsive Webdesign
\item Progressive Enhancement
\item Git
\item Netbeans
\item LAMP or similar for your operating system
\end{itemize}

As said in section \nameref{platforms}, the system is developed in a way, so that it should work on multiple platforms. 
This makes it sometimes difficult to describe certain installation processes in a way that would work for everybody. As
it's often most problematic, to get some Linux software running on Windows, we will mostly concentrate on the way those
things are done on this platform and give the instructions for other operating systems as an aside if necessary.

\section{Development Environment}

\subsection{Git}
% note from dominik: I changed some things to the actual latex commands, so that we can style later on if we want to 
The version control of all parts of the unplagged project is managed through Git. Since 2005, Git got more and more 
famous and many developers prefer it over Subversion because of its simplicity. However, nobody of our team ever used 
Git before so it was a challenge to get it running on all the systems. But we took the challenge to explore all the 
features Git offers. It is so much easier to create different branches and merge them again, than it is with other 
version control softwares like Subversion.

If you didn't use git before, you probably should watch this 8 minutes Git introduction video first:
\url{http://www.youtube.com/watch?v=RDGzF2M-zlo}

\subsubsection{Installating the Git Bash}
First of all let's get started with an introduction of how to install the Git console application, called Git Bash. 
Unfortunately all the GUIs we were evaluating didn't work as expected, so we decided to use it from the console only. 
A very good instruction on how to install the Git Bash can be found on the website of the github project:

\begin{description}
\item[Mac OS X:] \url{http://help.github.com/mac-set-up-git/}
\item[Windows:] \url{http://help.github.com/win-set-up-git/}
\item[Linux:] \url{http://help.github.com/linux-set-up-git/}
\end{description}

\subsubsection{Getting the source code of the unplagged project}
Now it is time to get the project source code on your machine. The whole unplagged project is hosted on github, so 
first you need to create an account on https://github.com. And then go to the directory where the project shall be 
located. An example for Mac~OS~X:

\begin{lstlisting}[caption=Cloning a repository]
cd ?Sites/unplagged.local?
git clone ?https://<usrname>@github.com/benoertel/unplagged.git?
\end{lstlisting}

\subsubsection{The most important git commands}

You are ready to use Git! Here are some more instructions on the most important commands and how to properly use it. 
However, if the given instructions in this manuall are not enough, feel free to checkout the whole Git manual on: 
\url{http://schacon.github.com/git/user-manual.html}

The unplagged project consists of several branches, which are used to develop and store code indepdently of the other 
developers. Once a new feature is done, it is merged into the master branch. The master branch usually includes only 
fully tested and deployable source code. 

As a new developer, it is important to create an own branch before doing anything else and switch into it.

\begin{lstlisting}[caption=Creating branches]
git branch mynewfeature
git checkout mynewfeature
\end{lstlisting}

Now anything in the repository can be changed, at any point changes can stored in the repository by using the git commit 
command. If new files were created, git add has to be executed as well.

\begin{lstlisting}[caption=Creating branches]
git add .
git commit -m "A message that describes the changes."
\end{lstlisting}

When the feature is fully working and approved, it has to be merged back to the master branch, in order to get deployed 
to the staging environment. To do this, the master branch has to be checked out, updated with git pull and then all changes
have to be merged from the new feature into the master branch and the feature branch has to be removed.

\begin{lstlisting}[caption=Creating branches]
git checkout master
git pull
git merge mynewfeature
git branch -d mynewfeature
\end{lstlisting}

In comparison to Subversion, for example Git has one more step to really write back to the repository. After a commit, 
a push has to be executed, each push can include multiple commits.

\begin{lstlisting}[caption=Creating branches]
git push
\end{lstlisting}

This is it, the changes to the repository have been pushed to the master branch.

\subsubsection{Handling conflicts in merging process}
It is possible, if two developers were working on the same part of  file, that a conflict raises during the merge. Such 
a conflict could look like this:

\begin{verbatim}
CONFLICT (content): Merge conflict in readme.txt

To https://github.com/benoertel/unplagged.git
 ! [rejected]        master -> master (non-fast-forward)
error: failed to push some refs to 'https://github.com/benoertel/unplagged.git'
To prevent you from losing history, non-fast-forward updates were rejected
Merge the remote changes (e.g. 'git pull') before pushing again.  See the
'Note about fast-forwards' section of 'git push --help' for details.

\begin{verbatim}
# Unmerged paths:
#   (use "git add/rm <file>..." as appropriate to mark resolution)
#
#    both modified:      readme.txt
#
\end{verbatim}

To reslove the issues, open the files listed in the error message, in this case "readme.txt" and decide how the correct 
version should look like, by removing all the "< < < < < < <  HEAD" and "> > > > > > > b478801d68267ef479acc5ca54544634c52c545c" 
parts.

\begin{verbatim}
<<<<<<< HEAD
The goal of this project is the creation of an easy-to-use, web-based
system to document and detect plagiarism in scientific papers.

hello world
=======

The goal of this project is the creation of an easy-to-use, web-based
system to document and detect plagiarism in scientific papers.

>>>>>>> b478801d68267ef479acc5ca54544634c52c545c
Just a change for educational purposes.
\end{verbatim}

Should look like this after merging:

\begin{lstlisting}[caption=Creating branches]
The goal of this project is the creation of an easy-to-use, web-based
system to document and detect plagiarism in scientific papers.

hello world

Just a change for educational purposes.
\end{lstlisting}

\subsection{Netbeans}

\subsection{Staging and Preview System}

This subsection will describe how to configure a virtual host properly. A virtual host is a domain that is mapped to the local web server. It is assumed that Apache, MySQL and PHP are already running on the machine. If not, here are some tutorial to get them all running:

Windows: \url{http://www.apachefriends.org/de/xampp-windows.html#1098}

Mac OS:\\
\url{http://www.djangoapp.com/blog/2011/07/24/installation-of-mysql-server-on-mac-os-x-lion/} \url{http://www.quarkstar.at/index.php/2009/05/18/webserver-aktivieren-und-konfigurieren-in-mac-os-x/}

The first step is to add the virtual host to the vhost config: 

\begin{lstlisting}[caption=Mac OS X: Creating virtual host]
sudo vi /private/etc/hosts
#add the following line:
"127.0.0.1 unplagged.local"

sudo vi /private/etc/apache2/extra/httpd-vhosts.conf
\end{lstlisting}

\begin{lstlisting}[caption=Windows: Creating a virtual host]
open C:\WINDOWS\system32\drivers\etc\hosts
#add the following line:
"127.0.0.1 unplagged.local"

open C:\xampp\apache\conf\httpd.conf
#Uncomment the following line
#Include conf/extra/httpd-vhosts.conf
open C:\xampp\apache\conf\extra\httpd-vhosts.conf
\end{lstlisting}

Add the following configuration to the httpd-vhosts.conf file:

\begin{lstlisting}[caption=Apache configuration]
<VirtualHost *:80>
ServerName unplagged.local
DocumentRoot "/Users/me/Sites/unplagged.local/public" 
SetEnv APPLICATION_ENV "development" 
<Directory /Users/benjamin/Sites/unplagged.local/public>
Options +Indexes +FollowSymLinks +ExecCGI
DirectoryIndex /index.php
AllowOverride All
Order allow,deny
Allow from all
</Directory>
</VirtualHost>
\end{lstlisting}

\section{Installation}
\subsection{Tesseract}
\subsection{Simtext}
\subsection{Imagemagick}



\section{Architectural Goals}

\subsection{Progressive Enhancement}

\subsection{Test Driven Development}

\subsection{Responsive Design}

\begin{appendix}

\chapter{Meetings}\label{ch:Meetings}
The following tables show the minutes of most of the team meetings.

\begin{figure}[htbp]
  \centering
    \includegraphics[width=\textwidth]{images/a_meetings/meeting_6}
  \caption{Meeting minutes no. 6}
  \label{fig:meeting minutes no. 6}
\end{figure}

\begin{figure}[htbp]
  \centering
    \includegraphics[width=\textwidth]{images/a_meetings/meeting_8}
  \caption{Meeting minutes no. 8}
  \label{fig:meeting minutes no. 8}
\end{figure}

\begin{figure}[htbp]
  \centering
    \includegraphics[width=\textwidth]{images/a_meetings/meeting_10}
  \caption{Meeting minutes no. 10}
  \label{fig:meeting minutes no. 10}
\end{figure}

\begin{figure}[htbp]
  \centering
    \includegraphics[width=\textwidth]{images/a_meetings/meeting_11}
  \caption{Meeting minutes no. 11}
  \label{fig:meeting minutes no. 11}
\end{figure}

\begin{figure}[htbp]
  \centering
    \includegraphics[width=\textwidth]{images/a_meetings/meeting_12}
  \caption{Meeting minutes no. 12}
  \label{fig:meeting minutes no. 12}
\end{figure}

\begin{figure}[htbp]
  \centering
    \includegraphics[width=\textwidth]{images/a_meetings/meeting_13}
  \caption{Meeting minutes no. 13}
  \label{fig:meeting minutes no. 13}
\end{figure}

\begin{figure}[htbp]
  \centering
    \includegraphics[width=\textwidth]{images/a_meetings/meeting_15}
  \caption{Meeting minutes no. 15}
  \label{fig:meeting minutes no. 15}
\end{figure}

\begin{figure}[htbp]
  \centering
    \includegraphics[width=\textwidth]{images/a_meetings/meeting_17}
  \caption{Meeting minutes no. 17}
  \label{fig:Meeting minutes no. 17}
\end{figure}

\begin{figure}[htbp]
  \centering
    \includegraphics[width=\textwidth]{images/a_meetings/meeting_18}
  \caption{Meeting minutes no. 18}
  \label{fig:Meeting minutes no. 18}
\end{figure}

\begin{figure}[htbp]
  \centering
    \includegraphics[width=\textwidth]{images/a_meetings/meeting_19}
  \caption{Meeting minutes no. 19}
  \label{fig:Meeting minutes no. 19}
\end{figure}

\chapter{Logged Time As Of March 22, 2012}

The following tables are some example reports generated from the logged time in Redmine. To find the most recent version
of these reports or to generate custom data analysis you can use the \enquote{Report} tool found in Redmine on the \enquote{Overview}
page.

% should be updated later on, now simply to make sure it works
% generate various reports in redmine and export as csv
% don't forget to escape special characters like _ or # in the input .csv, e. g. \_ or \#

\DTLloaddb{overview}{data/timelog-overview.csv}
\begin{table}[htbp]
  \caption{Overview By Member and Month}
  \centering
  \DTLdisplaydb{overview}
\end{table}

\begin{landscape}

\DTLsetseparator{,}

\DTLloaddb{issueMember}{data/timelog-issue-member.csv}
  \centering
 \DTLdisplaylongdb[caption=Overview By Member and Issue]{issueMember}

\DTLloaddb{sprints}{data/timelog-sprints.csv}
\begin{table}[htbp]
  \caption{Overview By Sprints}
  \centering
  \DTLdisplaydb{sprints}
\end{table}

\end{landscape}

\chapter{Mockups}\label{appendix:mockups}

\section{Hand-Drawn}

\begin{figure}[!h]
  \centering
    \includegraphics[width=\textwidth]{mockups/m_compare_result.jpg}
  \caption{Mockup – Compare results – digitalized }
  \label{fig:mCompareResultsMockup}
\end{figure}

\begin{figure}[!h]
  \centering
    \includegraphics[width=\textwidth]{mockups/m_media_list.jpg}
  \caption{Mockup – Media list – digitalized }
  \label{fig:mMediaListMockup}
\end{figure}

\begin{figure}[!h]
  \centering
    \includegraphics[width=\textwidth]{mockups/m_new_case.jpg}
  \caption{Mockup – New case – digitalized }
  \label{fig:1newCaseMockup}
\end{figure}

\begin{figure}[!h]
  \centering
    \includegraphics[width=\textwidth]{mockups/m_new_fragment.jpg}
  \caption{Mockup – New fragment – digitalized }
  \label{fig:1newCaseMockup}
\end{figure}

\begin{figure}[!h]
  \centering
    \includegraphics[width=\textwidth]{mockups/m_new_project.jpg}
  \caption{Mockup – New project – digitalized }
  \label{fig:mNewProjectMockup}
\end{figure}

\clearpage
\section{Digitalized}

\begin{figure}[htbp]
  \centering
    \includegraphics[width=0.86\textwidth]{mockups/1_new_case.png}
  \caption{Mockup – New case – digitalized }
  \label{fig:1newCaseMockup}
\end{figure}

\begin{figure}[!h]
  \centering
    \includegraphics[width=\textwidth]{mockups/2_list_fragments.png}
  \caption{Mockup – List fragments – digitalized }
  \label{fig:2listFragmentsMockup}
\end{figure}

\begin{figure}[!h]
  \centering
    \includegraphics[width=\textwidth]{mockups/3_new_fragment.png}
  \caption{Mockup – New fragment – digitalized }
  \label{fig:3newFragmentMockup}
\end{figure}

\begin{figure}[!h]
  \centering
    \includegraphics[width=0.97\textwidth]{mockups/4_show_fragment_for_approval.png}
  \caption{Mockup – Show fragment for approval – digitalized }
  \label{fig:4showFragmentForApprovalMockup}
\end{figure}

\begin{figure}[!h]
  \centering
    \includegraphics[width=\textwidth]{mockups/5_new_report.png}
  \caption{Mockup – New report – digitalized }
  \label{fig:5newReportMockup}
\end{figure}

\end{appendix}


\bibliography{biblio}

\end{document}