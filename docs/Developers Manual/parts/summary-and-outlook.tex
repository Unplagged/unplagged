\chapter{Summary and Outlook}\label{chap:summaryAndOutlook}

During our first project semester, we already immerged into the field of plagiarism detection very deeply. After
we read about exisiting websites and talked to Prof. Dr. Weber-Wulff about her experience and missing features
at VroniPlag, we defined our list of requirements. In the next step we developed first user interface mockups and
a basic layout, before we started the development of some requirements. 

The application of a modified version of Scrum as our agile software development method was a very interesting and new
experience to all of us. With the use of Redmine for keeping track of all issues, repository changes and time logs, everyone in the team was at anytime able to take a look on the current project state.

All in all the coditions for a succesful development were allocated properly and we already have implemented many features of the requirements list, that can be improved in the next semester.

Though the employed development processes in this semester were not as consistent as they should have been. The in theory defined rules, how to programm and how to test source code properly, were not always applied. So one of the main goals for the second semester is to improve this proccess. We need to focus on test driven development and to increase our velocity during the sprints, in order to get stable code more easily. 

Another issue we need to work on is the staging environment we are using. Some features behave differently on Mac OS and Windows machines. Since we have only one staging environment, which updates on every commit to our git repository, it sometimes happens that a feature working on Windows crashes the staging environment. As a solution we should setup a second pre-staging envirnoment which updates at every commit to the repository automatically and another more stable server, which gets updated manually as soon as the pre-staing environment is testet properly. 

The documentation of our work, which is represented in this developers manual was done in the end of the semester. In the next semester we are planning to extend and keep the manual up to date at every sprint, which means preferably more time for development and less time for documentation in the end.
