\chapter{Summary and Outlook}\label{chap:summaryAndOutlook}

During the first project semester, we already immerged into the field of plagiarism detection very deeply. After
we read about exisiting websites and talked to Prof. Dr. Weber-Wulff about her experience and missing features
at VroniPlag, we at least got an idea, how a software system for this domain could look like. After this phase of 
initial research, we 
developed first 
user interface mockups and
a basic layout, before we started the development on some user stories. 

The application of a modified version of Scrum as our agile software development method was a very interesting and new
experience to all of us, but also took us some more time than expected to figure out. But with the use of Redmine for 
keeping track of all 
issues, repository changes and time logs, 
everyone in the team is now able to take a look on the current project state anytime.

All in all, the conditions for a successful development were allocated properly and we already have implemented some 
features of the product backlog, that will be built upon in the next sprints.

Problematic was, that the employed development processes were not as consistent as they should have been. The in 
theory defined rules on how to program and how to test source code properly were not always applied. So one of the 
main goals for the next block of sprints is to improve this proccess. We need to focus on test driven development and 
increase our velocity during the sprints, in order to get stable code more easily. Maybe implementing some kind of 
review process would also help in solving those problems.

Another thing we want to work on is the staging environment we are using. Some features behave differently on Mac 
OS and Windows machines. Since we have only one staging environment, which updates on every commit to our git 
repository, it sometimes happens that a feature working on Windows crashes the staging environment. This is somehow 
nice, because those errors are encountered early, if the preview area is checked properly, but sometimes those problems
can be overlooked. As a solution 
we are thinking about setting up a second pre-staging environment which updates at every commit to the repository 
automatically and 
another more stable server, which can be updated manually via a script as soon as the pre-staging environment is tested properly. 

Even though we wrote down some information about the development process inside the Wiki of Redmine, most of the 
documentation as 
represented in this document had to be 
done in the end of the semester and not continuously during the process. 
In the next semester we are planning to extend and keep the manual up to date at every sprint, which means preferably 
more time for development and less time for documentation in the end.

We hope that this document was helpful to you! Please contact us if you have any questions or remarks.

\large 
\textbf{Best regards, \\
The Unplagged Team}