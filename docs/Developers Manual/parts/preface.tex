\chapter*{Introduction}

Even though the big media coverage and interest in plagiarism in Germany has very much subsided 
after Minister Guttenberg had to resign, because of the plagiarisms found in his doctoral thesis\citep[page 12]{Google2012}, 
the initial idea for the 
creation of the \enquote{Unplagged} project, whose development approach will be described here, can be found in this 
very case of plagiarism. Related to it were the formation of the
\href{http://de.guttenplag.wikia.com/wiki/GuttenPlag\_Wiki}{GuttenPlag} and it's descendent 
\href{http://de.vroniplag.wikia.com/wiki/Home}{VroniPlag}. Both are Wiki-based communities that are collaboratively 
discovering and collecting plagiarism in their respective cases and are kind of the role models for the way the
Unplagged system is developed.

The initial project idea and context were provided by our professor Dr. Debora Weber-Wulff and the two term master project
every media informatics student at the \href{http://htw-berlin.de/}{HTW-Berlin} has to take. As professor Weber-Wulff is 
a well known expert in Germany on the
topic of plagiarism,
does research in this field for over ten years and is actively involved in the VroniPlag
community\citep{Spiegel-Online2011}, she came up with the idea to build a dedicated system --- a \enquote{Plagiarism Detection Cockpit}\citep{Weber-Wulff2011} --- 
that is modeled after the experiences that were made
with the workflow used in VroniPlag and GuttenPlag.

So, to put it in a catchy marketing phrase, here is what Unplagged aims to become: 

\begin{quote}
\textbf{Unplagged is a simple, web-based, collaborative system to help discovering, collecting and 
documenting plagiarism in scientific papers.}
\end{quote}

To make things a bit more conceivable, we also often refer to it as a mixture of a very specialised text editor, with a focus on 
comparing texts and marking 
passages and a modern project management tool like \href{http://www.redmine.org/}{Redmine} or 
\href{http://www.atlassian.com/JIRA}{JIRA}, 
to manage the collaborative aspects of the system. The big distinction we make to other plagiarism software on the market is, 
that the approach is not to autodetect plagiarism, but focused on aiding the workflow of the users while  
searching for plagiarized
fragments inside a scientific paper, a homework assignment or any other kind of probable textual plagiarism.

This present document will be the handbook that gets you started if you are interested in helping us with the development of this 
open source project, which is licensed under the \href{http://www.gnu.org/licenses/quick-guide-gplv3}{GNU GPLv3}.

\section*{Chapter Overview}

One of the biggest problems we faced at the start was, that none of the team members had written a longer scientific
text than a bachelors
thesis and therefore the experience we got with actual scientific writing was very limited and very specific to the 
field of computer science. We understand the ethical problems, that come with the betrayal of 
good scientific practice of plagiators, but we simply can not relate easily to the amount of work that has to be put into 
a Ph D., or be as 
passionate about plagiarism as Prof. Weber-Wulff always is, because we never experienced it ourselves.

That is why we had a lot of catching up to do on the most important history behind VroniPlag, the different types
of plagiarism, different citation styles and the research Prof. Weber-Wulff and others had already done on systems that try to 
help finding plagiarism. Chapter \ref{chap:plagOverview}, \nameref{chap:plagOverview}, will
give a brief overview of the most important topics to get you up to speed with the domain of the software, if you are
not already familiar with it.

Chapter \ref{chap:systemRequirements} will be the place, where the development process is described and a collection and description 
of the parts of the system, that already exist or that we identified as 
necessary parts of Unplagged will be given. As the system is developed with an agile project management style, this will be done
primarily based on the user stories.

If you know all those things already and simply want to get started working and coding, you should probably jump
to \nameref{chap:developingUnplagged}. This chapter will give the technical insights into the system, the basic installation
steps and all necessary tools for you as a developer.

\section*{Conventions}

To markup important words in the text, the following typgraphical conventions are used:

\begin{description}
\item \textit{Italic} \hfill \\
  First used technical terms
\item \texttt{Constant Width} \hfill \\
  Programm code, file names, paths
\item \textbf{\texttt{Bold Constant Width}} \hfill \\
  Variables that have to be changed by the user
\end{description}
