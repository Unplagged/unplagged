% I'm not totally sure, if this should be described together or seperate, but somehow or SCRUM approach and the Redmine
% stuff should be mentioned somewhere before the developing part and as this are probably only three pages or somethin
% it probably could be integrated here
\chapter{Project Workflow and  Requirements}
\label{chap:systemRequirements}

First of all, we've got a confession to make: Unplagged is like a big playground of new 
workflows and technologies for us, as we are aiming to incorporate 
\enquote{best-practices} wherever possible, or at least what we currently consider to be best-practices. 

We believe this approach is necessary, because of the 
fact, that we are essentially trying to incubate Unplagged as a real open source project and this will 
only work if
it is well crafted and if cutting-edge workflows and technologies are used. Nearly all of the team members
are also working in some kind of web related side job, so we all got enough experiences with the problems that can 
occur during the maintenance of badly designed software.

Most of the times this works pretty well, but sometimes we are still trying to figure out how to 
get everyone up to speed with every technology and part of the system or how to divide the responsibilites carefully.

To start this project, we opted to use \textit{Scrum}\footnote{A nice introduction into Scrum is \enquote{The Scrum Primer} 
of the Scrum Alliance: \url{http://www.scrumalliance.org/resources/339}} 
as our agile development approach. If you are familiar with this
methodology, you may notice, that there could be a few problems when considering, that the team is working mostly 
distributed
without a common office and with very different time tables for each of the members.

We struggled a bit to tweak the workflow that is required by Scrum to fit the situation we faced, but you will see in the
following what we came up with.

\section{The Workflow}

To make it possible to work efficiently together in this kind of environment, we chose to use
\href{http://www.redmine.org/}{Redmine} as our project management tool, which you can access under:

\begin{itemize}
\item \url{http://tickets.unplagged.com}
\end{itemize}

If you register there, an administrator should grant you access to the tickets and the wiki, so that you can participate
in solving the problems at hand.

\begin{figure}[htbp]
  \centering
    \includegraphics[width=\textwidth]{images/roadmap.png}
  \caption{Redmine Roadmap}
  \label{fig:roadmap}
\end{figure}

What we are doing there is to map every \textit{Sprint} and the \textit{Product Backlog} to Redmine's notion of \enquote{Version} and
every identified \textit{User Story} to an \enquote{Issue}.

You can see in figure \ref{fig:roadmap} the view of the roadmap, with the current sprint 5 designated to create this very
document at the bottom and a not very well filled product backlog at the top. 

Normally 
every identified user story that is not part of the current sprint should be in the product backlog (and we got plenty), 
but as we were still working
in our small group at this point, the hassle of filling in all the tickets seemed unnecessary. This is something that
will be fixed in the near future, so that you are able to see where the development is going.

Currently we are working mostly with four week long sprints, to overcome the problem that we are not working fulltime
on the tickets, which is something that scrum normally assumes. 

To have a nice statistical overview and more planning security for the \enquote{scrums}, it is required to log the time 
that was spent on an issue within redmine.

% Redmine description, Meetings, Debbie, Scrum Cards, Server, Website

\subsection{Product Owner --- \enquote{The Debbie Meetings}}


\begin{figure}[!h]
  \centering
    \includegraphics[width=0.8\textwidth]{images/2011-11-15-user-stories-6.jpg}
  \caption{Scrum Meeting}
  \label{fig:scrumming}
\end{figure}

To figure out the user stories we mostly rely on what we internally call \enquote{Debbie Meetings}.
Normally at the end of every sprint, the members of the team meet with Prof. Weber-Wulff, who we state to be our \textit{Product
Owner}. We simply sit down there and talk about what should be implemented in the next sprint and collect it on cards as
you can see in figure \ref{fig:userStories}.

We consider this to be just a temporary way of handling this, because we hope that when we eventually have a prototype 
that we consider to have enough \enquote{business value} to be shown to more people, the focus will shift away from 
Prof. Weber-Wulff to a more broader understanding of the product owner. 

\begin{figure}[!h]
  \centering
    \includegraphics[width=0.8\textwidth]{images/2011-11-15-user-stories-4.jpg}
  \caption{User Stories}
  \label{fig:userStories}
\end{figure}

This means, that we want to open up our ticket system to directly collect new user stories and gather feedback for 
example from the VroniPlag group, which can be reached through their IRC channel \texttt{\#VroniPlag} on:

\begin{itemize}
\item \url{http://webchat.freenode.net/}
\end{itemize} 


\subsection{Team Meetings}

The Unplagged version of the \textit{Daily Scrum} is currently a weekly meeting. Even our meetings weren't daily, it was still a meeting which was respecting the other characteristics of a \textit{Daily Scrum}. Each team member used to say what he did for the last week and what he attends to do for the next week. These meetings were good occassions to verify the evolution of the project and determine the possilbe bottlenecks as soon as possible. 

\minisec{}
Our projectteam was doing the meetings every monday for at least one hour. Most of the times we met at the same place, if due to appointment collisions we couldn't manage to meet in persons, we met in Skype. So we never used to skip the weekly meeting.
\minisec{}
For each meeting that we held someone was responsible for writing the protocol about the meeting. For each protocol a new wiki-page was created to our \href{http://www.redmine.org/}{Redmine}. To put the protocols to our \href{http://www.redmine.org/}{Redmine} was really helpful for the teamwork. So everybody had access to it and could if necessary have a look on the decisions which were made in the past.
\minisec{}
During the meetings at least these following points have been discussed:
\minisec{}
\begin{itemize}
\item What was done last week. 
\item What will be done for the next week.
\item Organisation issues.
\item External dependencies issues.
\item Problems encountered last week.
\item Further questions.
\end{itemize}
\minisec{}
All the protocols which were made during the phase one are available on the appendix \ref{ch:minutes} (page number \pageref{ch:minutes}).

\section{Target Group}

The main goal of the application is to support the detection of plagiarism. So the target group is composed of people who want to correct documents and be sure that this does not contain any plagiarism.
\minisec{}
The following list is showing people who could be interested in using this application:
\minisec{}
\begin{itemize}
\item Professors 
\item Teachers
\item Academics
\item Instructors
\item Tutors
\item ...
\end{itemize}
\minisec{}
This people are mostly not it-specialists, so it is important to focus on it during the development of the application. This has an influence on the way they will use the application: it must be easy and naturally to use, the results must be easy to understand and reliable.

\section{User roles}
%This is copied from the wiki, so probably copy back there when the text is finished
As the Unplagged system will provide a permission based user system, our goal is to make it possible, to create custom user roles from an administration area and make it possible for users to have multiple user roles in one case and also different roles for different cases.
The standard roles which will be provided by the system are:

\begin{description}
\item[Guest]
A user without a valid login can only see the parts of cases that are set to be public.
\item[Registered]
Registered users can get \enquote{promoted} to higher roles and contribute to publicly editable cases.
%Note: Do we want users to register themselves? Or do we have just an admin, that takes care of this? I'm currently also not sure if this will really need to be a role, or if Guest and Registerd are just two basic states.
\item[Collaborator]
Collaborators are registered users who were granted access to a specific case. Collaborators can access and edit these projects.
\item[Case-Manager]
Case-Managers can set up new cases and manage colloborators for their cases and project versions. They may have the permission to add or remmove project members.
%Note: What does manage project versions mean?
\item[Admin]
An admin owns all permissions, such as user administration or project administration. They also hvae the ability to to block/unblock an existing case.
\end{description}

\section{Basic functionalities}
The following points are showing the basic functionalities of the application.
\subsection{Register}
A person who want to use the application need to register first. It gets the abbility to do it, as soon as it opens the start-page and hit the "Jetzt anmelden" link.
\begin{figure}[!ht]
  \centering
    \includegraphics[width=0.97\textwidth]{images/basic_functionalities/reg1.jpg}
  \caption{register1}
  \label{fig:register1}
\end{figure}

Below you see the register form which has to be filled.
\begin{figure}[!ht]
  \centering
    \includegraphics[width=0.97\textwidth]{images/basic_functionalities/register_form.jpg}
  \caption{register1}
  \label{fig:register1}
\end{figure}

Every entry here will be saved in the database.

\subsection{Login}

The login functionality is an easy to use form, which can be successfully used from every user which is already registered. With a click of the "Einloggen" link, the login form will be visible.
\begin{figure}[!ht]
  \centering
    \includegraphics[width=0.97\textwidth]{images/basic_functionalities/oderEinloggen.jpg}
  \caption{"Einloggen" Link}
  \label{fig:einloggen}
\end{figure}

The received login datas that are taken from the login form will be compared with the entries in the database, are the entries valid, the user has the right to use further functions in the application.

\begin{figure}[!ht]
  \centering
    \includegraphics[width=0.97\textwidth]{images/basic_functionalities/login_form.jpg}
  \caption{login form}
  \label{fig:einloggen}
\end{figure}

\begin{figure}[!ht]
  \centering
    \includegraphics[width=0.97\textwidth]{images/basic_functionalities/after_login.jpg}
  \caption{after login view}
  \label{fig:einloggen}
\end{figure}
\subsection{Files overview}
The tab "Dateien" is showing the list of files which were added from an user. Not only shows it an overview it also has several additionaly functions which can be processed for a file.
\begin{figure}[!ht]
  \centering
    \includegraphics[width=0.97\textwidth]{images/basic_functionalities/dateien.jpg}
  \caption{overview of files}
  \label{fig:einloggen}
\end{figure}
\subsubsection{Set target}
If the user wants to mark a specific file as his target file, he can do it just with a click on the following symbol.
\begin{figure}[!ht]
  \centering
    \includegraphics[width=0.05\textwidth]{images/basic_functionalities/set_target2.jpg}
  \caption{target function symbol}
  \label{fig:einloggen}
\end{figure}
\begin{figure}[!ht]
  \centering
    \includegraphics[width=0.97\textwidth]{images/basic_functionalities/set_target1.jpg}
  \caption{not marked file}
  \label{fig:einloggen}
\end{figure}
Afterwards a binoculars icon will be shown.
\begin{figure}[!ht]
  \centering
    \includegraphics[width=0.97\textwidth]{images/basic_functionalities/set_target3.jpg}
  \caption{marked target file}
  \label{fig:target file}
\end{figure}
\subsubsection{Parse}
The parse function parses the mentioned file and forwards the parsed file to the "Dokumente" section.
Witht a click of the following symbol, the parse function will be fired.
\begin{figure}[!ht]
  \centering
    \includegraphics[width=0.05\textwidth]{images/basic_functionalities/parse_symbol.jpg}
  \caption{parse symbol}
  \label{fig:parse symbol}
\end{figure}
\subsubsection{Download}
With the disk-symbol the user can download the listed file.
\begin{figure}[!ht]
  \centering
    \includegraphics[width=0.05\textwidth]{images/basic_functionalities/download_symbol.jpg}
  \caption{download symbol}
  \label{fig:download symbol}
\end{figure}
\subsubsection{Delete}
To delete a file the user only needs to click on the following symbol.
\begin{figure}[!ht]
  \centering
    \includegraphics[width=0.05\textwidth]{images/basic_functionalities/delete_symbol.jpg}
  \caption{delete symbol}
  \label{fig:delete symbol}
\end{figure}
After the function was fired the path to the file in the database and the file in the related folder will be deleted.
\subsubsection{File upload}
The file uploader is a important function, it gives the user the opportunity to upload data which he want to work on it. Also has the user the opportunity to give the file a desired name.
\begin{figure}[!ht]
  \centering
    \includegraphics[width=0.97\textwidth]{images/basic_functionalities/datei_hochladen.jpg}
  \caption{file uploader}
  \label{fig:file uploader}
\end{figure}

\subsection{Parsed files overview}
So after a file was parsed with the parsed function from the "Dateien" section, it will be listed in the "Dokumente" tab.
\subsubsection{Integration of detection plagiarism software}
The "plag-detection" functionality gives user the possibility to check a parsed file of plagiarism automatically. It uses the application interface from the company "PlagAware". This is possible during our coorporation with "PlagAware".
With the following symbol the "plag-detection" function can be used on the file.The detailed result report is not to see on the application. But the user has the possibility to see it on the "PlagAware" page.The user can get several informations on the application, like the success of the scanning and the procentaged part of the plagiarism.
\begin{figure}[!ht]
  \centering
    \includegraphics[width=0.05\textwidth]{images/basic_functionalities/plagdetection_symbol.jpg}
  \caption{detect plagiarism symbol}
  \label{detect plagiarism symbol}
\end{figure}
\subsubsection{Delete}
To delete a parsed file the user only needs to click on the following symbol.
\begin{figure}[!ht]
  \centering
    \includegraphics[width=0.05\textwidth]{images/basic_functionalities/delete_symbol.jpg}
  \caption{delete symbol}
  \label{fig:delete symbol}
\end{figure}
After the function was fired the path to the file in the database and the file in the related folder will be deleted.
\subsubsection{Simtex}
The simtext function is meant to compare two files with each other. So equal parts in texts can be detected.
\begin{figure}[!ht]
  \centering
    \includegraphics[width=0.97\textwidth]{images/basic_functionalities/simtext_running.png}
  \caption{e.g. simtext running}
  \label{fig: simtext running}
\end{figure}
The simtext function also creates a report, where further detailed information can be seen.
\begin{figure}[!ht]
  \centering
    \includegraphics[width=0.97\textwidth]{images/basic_functionalities/simtext_report.jpg}
  \caption{e.g. simtex report}
  \label{fig:e.g. simtex report}
\end{figure}
In the report the equal parts on the compared files are shown.
\subsection{Edit profile}
The "Profil bearbeiten" form is a form which gives the user the opportunity to change profile infromations that he gave before. All edit and saved entries will be saved in the database.
\begin{figure}[!ht]
  \centering
    \includegraphics[width=0.97\textwidth]{images/basic_functionalities/edit_profile.jpg}
  \caption{profile edit form}
  \label{fig: profile edit form}
\end{figure}
\subsection{Googlesearch}
The "Googlesearch" function is content from the contextmenu of the application. The contextmenu only shows if the user marked text on the application. So the user can direclty search for text in google from the application. After the user clicked on "GoogleSuche nach:" the browser will open the google page with the results of the search. To delete the searchwords the user have to click on "Goolge-Suchwörter löschen".
\begin{figure}[!ht]
  \centering
    \includegraphics[width=0.97\textwidth]{images/basic_functionalities/contextmenu.jpg}
  \caption{googlesearch contextmenu}
  \label{fig:contextmenu}
\end{figure}

\section{Document Parser}

\section{Detection Modes}

\section{Plugin Architecture}

%Perhaps better on the way for every part?
\section{Use Cases}
The following diagramms are so called "uml use case diagramms". The use cases are ment to visualize the interaction between the user and the system.
\subsection{Register}
The not registered actor wants to use the application. To use the application he have to first register. Which includes the follow steps:
\minisec{}
\begin{itemize}
\item User opens the webpage of unplagged.
\item User clicks on the "Jetzt anmelden" link.
\item User puts needed information in the register form.
\item User saves entries.
\end{itemize}
\begin{figure}[!ht]
  \centering
    \includegraphics[width=0.97\textwidth]{images/use_cases/registration.png}
  \caption{use case register}
  \label{fig:use case register}
\end{figure}

\subsection{Login}
A registered user wants to work at the application.
\minisec{}
Condition: user registered.
\minisec{}
\begin{itemize}
\item User opens the webpage of unplagged.
\item User clicks on the "Einloggen" link.
\item User puts needed information in the login form.
\item User sends entries.
\end{itemize}
\begin{figure}[!ht]
  \centering
    \includegraphics[width=0.97\textwidth]{images/use_cases/login.png}
  \caption{use case login}
  \label{fig:use case login}
\end{figure}

\subsection{File Upload}
A user wants to upload a file.
\minisec{}
Condition: user logged in and opend the tab "Dateien".
\minisec{}
\begin{itemize}
\item User clicks on the button "Datei hochladen".
\item User browses for the file which he want to upload.
\item (Optional: User gives new name for the uploaded file.)
\end{itemize}

\subsection{Parse}
A user wants to parse a file.
\minisec{}
Condition: user already uploaded the file successfully on the section "Dateien".
\minisec{}
\begin{itemize}
\item User clicks on the parse-icon button.
\end{itemize}
\subsection{Simtext}
\subsection{Googlesearch}
A user wants to google-search a word in the application.
\minisec{}
\begin{itemize}
\item User clicks on a word or text.
\item User makes a right click on the marked part.
\item User clicks on the context menu of the  "GoogleSuche nach:" link.
\item User can see results in extern browser page.
\end{itemize}


