% I'm not totally sure, if this should be described together or seperate, but somehow or SCRUM approach and the Redmine
% stuff should be mentioned somewhere before the developing part and as this are probably only three pages or somethin
% it probably could be integrated here
\chapter{Project Workflow and  Requirements}\label{chap:systemRequirements}

First of all, we've got a confession to make: Unplagged is like a big playground of new 
workflows and technologies for us, as we are aiming to incorporate 
\enquote{best-practices} wherever possible, or at least what we currently consider to be best-practices. 

We believe this approach is necessary, because of the 
fact, that we are essentially trying to incubate this as a real open source project and this will 
only work if
it is well crafted and if cutting-edge workflows and technologies are used. Nearly all of the team members
are also working in some kind of web related side job, so we all got enough experiences with the problems that can 
occur during the maintenance of badly designed software.

Most of the times this works pretty well, but sometimes we are still trying to figure out how to 
get everyone up to speed with every part of the system or how to divide the responsibilites carefully.

To start this project, we opted to use \textit{Scrum} as our agile development approach. If you are familiar with this
methodology, you may notice, that there could be a few problems when considering, that the team is working mostly distributed
without a common office and with very different time tables for each of the members.


To make it possible to work efficiently together in this kind of environment, 
\href{http://www.redmine.org/}{Redmine} as it's project management tool, which you can access under:

\begin{itemize}
\item \url{http://tickets.unplagged.com}
\end{itemize}

If you register there, an administrator should grant you access to the tickets and the wiki, so that you can participate
in solving the problems at hand. Our current workflow 

% Redmine description, Meetings, Debbie, Scrum Cards, Server, Website


\begin{figure}[!h]
    \includegraphics[width=0.8\textwidth]{images/2011-11-15-user-stories-6.jpg}
  \caption{Scrum Meeting}
  \label{fig:scrumming}
\end{figure}

\begin{figure}[!h]
    \includegraphics[width=0.8\textwidth]{images/2011-11-15-user-stories-4.jpg}
  \caption{User Stories}
  \label{fig:userStories}
\end{figure}

\section{Target Group}

\section{User roles}

\section{Basic functionalities}

\section{Document Parser}

\section{Detection Modes}

\section{Plugin Architecture}

%Perhaps better on the way for every part?
\section{Use Cases}
